\documentclass{article}
\usepackage{amsmath,amssymb}
\usepackage[numbers]{natbib}
\usepackage{geometry}
 \geometry{
 a4paper,
 total={170mm,257mm},
 left=30mm,
 top=30mm,
 right=30mm,
 bottom=30mm
 }
 
\usepackage{graphicx}
\usepackage{caption}
\usepackage{subcaption}
\usepackage{float}

\usepackage{multirow}
\usepackage{siunitx}
\usepackage{booktabs}
\usepackage{gensymb}

\usepackage{algorithm}
\usepackage{algpseudocode}

\usepackage[table]{xcolor}

\setlength{\parskip}{1em}

\newcommand{\citecustom}[1]{\citeauthor{#1} \cite{#1}}

\title{Reinforcement Learning Assignment-3 \\
	\Large Markov Decision Process and Dynamic Programming \\}
\begin{document}
\author{Utkarsh Prakash \\ \normalsize 180030042}
\maketitle
\section{Markov Decision Process}
A Markov Decision Process is a tuple $<\mathcal{S}$, $\mathcal{A}$, $\mathcal{P}$ and $\mathcal{R}>$ where $\mathcal{S}$ is the state space,
$\mathcal{A}$ is the action space, $\mathcal{P} : \mathcal{S} \times \mathcal{S} \times \mathcal{A} \rightarrow [0, 1]$ is the probability transition
function and $\mathcal{R} : \mathcal{S} \times \mathcal{S} \times \mathcal{A} \rightarrow$ $\mathbb{R}$ is the immediate reward function.
In order words, $\mathcal{P}(i, j, a) = P_{ij}(a)$ is the probability of transitioning from state $i$ to state $j$ when action $a$ is chosen.
Similarly, $\mathcal{R}(i, j, a)$ is the reward obtained on transitioning from state $i$ to state $j$ when action $a$ is chosen.

\section{Policy Iteration}
    \begin{algorithm}
        \caption{Policy Iteration}\label{policy_iteration}
        \begin{algorithmic}
            \State \textbf{input}: MDP $<\mathcal{S}$, $\mathcal{A}$, $\mathcal{P}$ and $\mathcal{R}>$, $\gamma$, $\epsilon$
            \State $\mu \gets [0,..., 0]$ \Comment{Initial Policy}
            \State $\mu' \gets [0,..., 0]$ \Comment{Temporary for storing present iteration's policy}
            \While{$\mu' \ne \mu$}
                \State $\mu = \mu'$
                \State
                \State \textbf{Policy Evaluation Step}
                \State $V^{0}_{\mu} \gets [0,..., 0]$
                \While{$\delta >= \epsilon$}
                    \State $\delta \gets 0$
                    \State 1. $V^{k+1}_{\mu}(i) = \sum_{j \in \mathcal{S}} P_{ij}(\mu(i)) [\mathcal{R}(i, j, \mu(i)) + \gamma V^{k}_{\mu}(j)] \forall i \in \mathcal{S}$
                    \State 2. $\delta = \max_{i \in \mathcal{S}}(|V^{k+1}_{\mu}(i) - V^{k}_{\mu}(i)|)$
                \EndWhile
                \State 
                \State \textbf{Policy Improvement Step}
                \State The value of $V^{k+1}_{\mu}$ in the last iteration of Policy Evaluation is $V_{\mu}$. Using this calculate $q_{\mu}(i, a) \forall i \in \mathcal{S}$ and $a \in \mathcal{A}$.
                \State $\mu'(i) = arg \max_{a \in \mathcal{A}} q_{\mu}(i, a) \forall i \in \mathcal{S}$
            \EndWhile
            \State The policy thus obtained is the optimal policy.
        \end{algorithmic}
    \end{algorithm}

\section{Value Iteration}
\begin{algorithm}[H]
    \caption{Value Iteration}\label{value_iteration}
    \begin{algorithmic}
        \State \textbf{input}: MDP $<\mathcal{S}$, $\mathcal{A}$, $\mathcal{P}$ and $\mathcal{R}>$, $\gamma$, $\epsilon$
        \State $V^{0} \gets [0,..., 0]$

        \While{$\delta >= \epsilon$}
            \State $\delta \gets 0$
            \State 1. $V^{k+1}(i) = \max_{a \in \mathcal{A}}\sum_{j \in \mathcal{S}} P_{ij}(a) [\mathcal{R}(i, j, a) + \gamma V^{k}(j)] \forall i \in \mathcal{S}$
            \State 2. $\mu(i) = arg \max_{a \in \mathcal{A}}\sum_{j \in \mathcal{S}} P_{ij}(a) [\mathcal{R}(i, j, a) + \gamma V^{k}(j)] \forall i \in \mathcal{S}$
            \State 2. $\delta = \max_{i \in \mathcal{S}}(|V^{k+1}(i) - V^{k}(i)|)$
        \EndWhile
        \State The policy thus obtained is the optimal policy.
    \end{algorithmic}
\end{algorithm}

\section{Grid World MDP}
Let's suppose the agent lives in the $4 \times 3$ environment as shown in Table. \ref{grid_world}. The reward that the agent gets in a particular
state is also indicated in the figure. In each of the state the agent needs to choose an action from \{Up, Down, Left, Right\}. The
agent is successful in reaching the state in which itends to reach by taking an action with probability 0.8 and reaches the states
perpendicular to the direction of action with the remaining probability (with both the perpendicular directions being equally-likely).
Let's suppose the top-leftmost cell is the origin of a coordinate system. Then according to this coordinates system, the cell at 
the coordinate (2, 2) is wall or a prohibited state. The discount factor $\gamma$ for the MDP is 0.9. Goal of the MDP is to 
maximize the expected total discounted reward.

\begin{table}[H]
    \begin{center}
    \renewcommand{\arraystretch}{2}
    \begin{tabular}{ | m{1cm} | m{1cm}| m{1cm} | m{1cm} | } 
      \hline
      0 & 0 & 0 & \cellcolor{green!25}1 \\ 
      \hline
      0 & \cellcolor{gray!50}Wall & 0 & \cellcolor{red!25}-100 \\ 
      \hline
      0 & 0 & 0 & 0 \\ 
      \hline
    \end{tabular}
    \caption{Grid World}
    \label{grid_world}
    \renewcommand{\arraystretch}{1}
\end{center}
\end{table}
	
\noindent %The next paragraph is not indented
We run the Policy Iteration and Value Iteration algorithm to obtain the optimal policy and optimal value function with $\epsilon = 1e-10$.
We choose the initial policy for Policy Iteration algorithm to be moving in the Up direction for all the states. The Policy 
Iteration algorithm took 3 iterations to converge whereas Value Iteration algorithm took 239 iterations to converge. The results 
obtained are shown as below:

\begin{table}[H]
    \renewcommand{\arraystretch}{2}
    \begin{minipage}{.5\textwidth}
        \begin{center}
        \begin{tabular}{ | c | c| c | c | } 
            \hline
            $\rightarrow$ & $\rightarrow$ & $\rightarrow$ & \cellcolor{green!25}$\uparrow$ \\ 
            \hline
            $\uparrow$ & \cellcolor{gray!50}Wall & $\leftarrow$ & \cellcolor{red!25}$\leftarrow$ \\ 
            \hline
            $\uparrow$ & $\leftarrow$ & $\leftarrow$ & $\downarrow$ \\ 
            \hline
        \end{tabular}
        \caption{Optimal Policy}
        \end{center}
    \end{minipage}%
    \begin{minipage}{.5\textwidth}
        \begin{center}
        \begin{tabular}{ | m{1cm} | m{1cm}| m{1cm} | m{1cm} | } 
            \hline
            5.47 & 6.31 & 7.19 & \cellcolor{green!25}8.67 \\ 
            \hline
            4.80 & \cellcolor{gray!50}Wall & 3.35 & \cellcolor{red!25}-96.67 \\ 
            \hline
            4.16 & 3.65 & 3.22 & 1.52 \\ 
            \hline
        \end{tabular}
        \caption{Optimal Value Function}
    \end{center}
    \end{minipage}%
    \renewcommand{\arraystretch}{1}
\end{table}
\noindent %The next paragraph is not indented
We can intuitively reason out why this should be the optimal policy. When the agent is in states close to the loosing state (state 
with -100 reward), the optimal action should be such that the probability of reaching the loosing state should be as low as 
possible (preferably 0). For other states, the optimal action should such that we reach the goal state (state with 1 reward) as soon
as possible because the present rewards are more valuable than the future rewards.

\section{Jack's Car Rental Problem}
\textbf{Example 4.2 from \citecustom{sutton2018reinforcement}}: Jack manages two locations for a nationwide car
rental company. Each day, some number of customers arrive at each location to rent cars.
If Jack has a car available, he rents it out and is credited \$10 by the national company.
If he is out of cars at that location, then the business is lost. Cars become available for
renting the day after they are returned. To help ensure that cars are available where
they are needed, Jack can move them between the two locations overnight, at a cost of
\$2 per car moved. We assume that the number of cars requested and returned at each
location are Poisson random variables, meaning that the probability that the number is
n is $\frac{\lambda^{n}}{n!}e^{-\lambda}$, where $\lambda$ is the expected number. Suppose $\lambda$ is 3 and 4 for rental requests at
the first and second locations and 3 and 2 for returns. To simplify the problem slightly,
we assume that there can be no more than 20 cars at each location (any additional cars
are returned to the nationwide company, and thus disappear from the problem) and a
maximum of five cars can be moved from one location to the other in one night. The discount factor $\gamma$ for the MDP is 0.9.
Goal of the MDP is to maximize the expected total discounted reward (rent). \par
	
\noindent %The next paragraph is not indented 
\textbf{Time Steps:} Days \\
\textbf{Actions:} The net numbers of cars moved between the two locations overnight. \\
\textbf{States:} The state is the number of cars at each location at the end of the day. 

\subsection{Policy Iteration}
Figure \ref{policy_iter_jack_problem} show the sequence of policies found by the Policy Iteration algorithm starting with the policy that no car
is moved between the two locations overnight and $\epsilon=1e-4$. The Policy Iteration algorithm took 4 iterations to converge
to the optimal policy.

\begin{figure}
    \graphicspath{ {../Experiments/JackRentalProblem/PolicyIteration/} }
    \begin{center}
    \includegraphics[width=15cm]{Compact1.png}
    \end{center}
    \caption{The sequence of policies found by the Policy Iteration algorithm along with value function for the final policy. The
    heatmaps in the first five figure show, for each number of cars at each location at the end of the day, the number of cars moved
    from first location to the second (negative numbers indicate transfers from the second location to the first). The darkest 
    colour indicates +5 car transfers and lightest colour indicates -5 car transfers.}
    \label{policy_iter_jack_problem}
\end{figure}

\subsection{Value Iteration}
Figure \ref{value_iter_jack_problem} show the sequence of policies found by the Value Iteration algorithm with $\epsilon=1e-4$. The Value Iteration
algorithm took 121 iterations to converge to the optimal value function.
\begin{figure}
    \graphicspath{ {../Experiments/JackRentalProblem/ValueIteration/} }
    \begin{center}
    \includegraphics[width=15cm]{Compact.png}
    \end{center}
    \caption{The sequence of policies found by the Value Iteration algorithm (only few of them included) along with value function
    for the final policy. The heatmaps in the first eight figure show, for each number of cars at each location at the end of the
    day, the number of cars moved from first location to the second (negative numbers indicate transfers from the second location 
    to the first). The darkest colour indicates +5 car transfers and lightest colour indicates -5 car transfers.}
    \label{value_iter_jack_problem}
\end{figure}

\noindent %The next paragraph is not indented 
We can intuitively understand why the optimal policy should be as above. In order to maximize our reward, the optimal policy should
be such the number of cars at each location are approximately equal. This is because, there is a cap on the number of cars
each location can have and the $\lambda$ (i.e. rate parameter of the Poisson distribution) is almost same for return and request at each
location. Therefore, we see in Figure \ref{policy_iter_jack_problem} and \ref{value_iter_jack_problem} above, that along the 
diagonal, where the number of cars at each location is same, the optimal policy is that we don't move any car. On the other hand, 
along the off-diagonal, where there is a skew in the numbers of cars at a location, the optimal policy is to move cars from one 
location to the other. 


\section{Conclusion}
\begin{itemize}
    \item In both the MDPs the Policy Iteration algorithm took fewer steps to converge than the Value Iteration algorithm.
    \item In the Jack's Car Rental Problem, we note that the Value Iteration algorithm converged very early (around iteration 20) 
    to the optimal policy but the algorithm continues because it has not converged to the optimal value function.
    \item In Policy Iteration algorithm, we observe that there is a strict improvement in the policy after every iteration.
\end{itemize}



\bibliographystyle{plainnat}
\bibliography{references.bib}


\end{document}